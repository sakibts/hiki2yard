\documentclass[12pt,a4paper]{jsarticle}
\usepackage[dvipdfmx]{graphicx}
\begin{document}
\section{TexManual}
\subsection{teXShopの設定}\begin{itemize}
\item 1 『設定』 ▶ ︎書類 ▶ ︎エンコーディング ▶︎ Unicode(UTF-8)
\item 2  『設定プロファイル』▶︎ pTeX(ptex2pdf)を選択
\item 3  TeXShopを再起動
\end{itemize}
\subsection{必要なgemの確認}\begin{itemize}
\item gem listコマンドを入力,以下のgemが入っているか確認\begin{itemize}
\item hikidoc
\item hikiutils
\end{itemize}
\end{itemize}
▶︎︎無ければ(sudo) gem install hogehogeで入れる

\subsection{入力の方法}\begin{itemize}
\item open -a mi hogehogeでファイルを開き,入力していく
\end{itemize}
\subsection{作ったtextをページにして表示}\begin{itemize}
\item hiki -e hogehoge で編集
\item hiki -u hogehoge で表示
\end{itemize}
\subsection{pdfファイルへの変換方法}\begin{itemize}
\item 1 hiki2latex hogehoge.hiki > hogehoge.tex
\item 2 hikidoc hogehoge.hiki > hogehoge.html (必要あるのか不明)
\item 3 open hogehoge.tex
\item 4TeXShopが開く. 左上のタイプセットを押すとpdfファイルが出力される
\end{itemize}
\end{document}
