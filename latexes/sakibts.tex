\documentclass[10pt,a4j,twocolumn]{jsarticle}

\usepackage[dvipdfmx]{graphicx}

\setlength{\textheight}{275mm}
\headheight 5mm
\topmargin -30mm
\textwidth 185mm
\oddsidemargin -15mm
\evensidemargin -15mm
\pagestyle{empty}
\begin{document}
\section{title}
yardによる数値計算関連文書作成システムの構築

\section{作業(16/8/4)}
以下の作業を今週中にやってください.
\begin{enumerate}
\item hiki2latexをhiki2yard.gemspecに入れて,bundle updateで取ってくる.
\item hiki2latexを動かしてみる.\begin{enumerate}
\item \verb|nishitani0のDocumentWriting(http://nishitani0.kwansei.ac.jp/~bob/nishitani0/Internal/DocumentWriting.html)|にある,hiki2latex, 手順・コツ,幾つかのformatひな形を読んでいれてみてください.
\end{enumerate}
\item Rakefileにsakibts.hikiをlatex/sakibts.texに変換するruby codeを書く.
\end{enumerate}
できないとみんなの中間発表用のabstractが作れません.がんばってね.

\section{幾つかの注意(16/8/17)}
updateされたのに対する幾つかの注意です.

\subsection{システムについての注意}\begin{itemize}
\item directoryのなかで,何処にfileを置くかは決めてあります.なぜかは概要に書きました.勝手にdirecotryを作るのではなく,今あるdirectoryの中での操作をRakefileに書き加えるようにしてください.\begin{itemize}
\item tex\_sampleは不用ですので,削除してください.
\end{itemize}
\item hikis/TexManual.hikiに移しました.そこから自動でlatexesへ生成するようにRakefileを修正しています.
\end{itemize}
\subsection{hikiの記述に関する注意}\begin{itemize}
\item number listは\#で書いていく.
\item へんな記号は使わない.右三角.これは,latexで解釈できないため,変換されないためです.
\item $\rightarrow$を使う.ただし,hikiでの表示ができていない.これは,hiki2yardではできません.\begin{itemize}
\item mathjaxを使えるようにします.その手順は次に記述します.
\end{itemize}
\end{itemize}
\subsection{mathjax-yardについて}
標準yardで変換されるmdファイル内で,数式を表示することを可能にする拡張機能です.yard処理の前後にmathjaxのための処理を挟んで,標準的なgemディレクトリ内にあるmdファイルをdoc/file.**に変換後にlatex記法の数式を表示します.

\subsubsection{操作法}\begin{quote}\begin{verbatim}
gem install mathjax-yard
\end{verbatim}\end{quote}
でinstall.
\begin{quote}\begin{verbatim}
mathjax-yard -i
\end{verbatim}\end{quote}
でinitしてくれます.後は, 
\begin{quote}\begin{verbatim}
rake myard
\end{verbatim}\end{quote}
でyardがhtmlに自動的に変換してくれます.
\end{document}
